\sectionTitle{Experience}{}
% 1 is end date
% 2 is title
% 3 is location
% 4 is NA
% 5 is start date
% 6 is description
\begin{experiences}
\researchexperience
    {May 2021}
    {École Polytechnique Fédérale de Lausanne | SCI-STI-MM Lab{\normalfont  ~ [\href{https://www.epfl.ch/labs/gramm/}{\small{\websiteSymbol}}]}}
    {Remote / Lausanne, Switzerland} {}
    {Jan 2021}
    {\textit{Research Intern (Bachelor Thesis) | Advisor:  \href{https://people.epfl.ch/marco.mattavelli?lang=en}{Prof. Marco Mattavelli}}\\
    Worked on the MPEG-G standard, specifically studying compression techniques to use for storing data.
    }
\emptySeparator

\researchexperience
    {Dec 2020}
    {NVIDIA}
    {Bangalore, India} {}
    {Aug 2020}
    {\textit{Hardware Engineer Intern | Manager:  \href{https://www.linkedin.com/in/ramanathan-sambamurthy-4540b9168}{Ramanathan Sambamurthy}}\\
    Worked on an internal development hardware debugging tool with a high-functioning CPU chip.
    }
\emptySeparator

\researchexperience
    {July 2019}
    {Brontominds}
    {Bangalore, India} {}
    {May 2019}
    {\textit{Software Engineer Intern | Mentor:  \href{https://in.linkedin.com/in/anupammittal}{Anupam Mittal}}\\
    Worked on creating and using a dataset, to evaluate items present in a diner menu, based on previous transactions.
    }
\emptySeparator

\researchexperience
    {July 2018}
    {Tata Communications Ltd.}
    {Pune, India} {}
    {May 2018}
    {\textit{Software Engineer Intern}\\
    Developed an automated internal tool to go through a periodically generated dataset about communication circuits across cities, search for faults, and make a report for each one according to a pattern.
    }
  
\end{experiences}
\vspace{-4mm}
