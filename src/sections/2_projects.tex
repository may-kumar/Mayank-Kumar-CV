\sectionTitle{Projects}{}
\begin{projects}

\project
	{Building a WSN for Precision Agriculture}{Jan'20 - May'20}
	{}
	{\begin{itemize}
	\setlength\itemsep{0.3em}
    \item Various motes are to be placed and used as a WSN for monitoring and tracking growth of a crop.
    \item The network stack has to be developed to make the network as dynamic, long-lasting and robust as possible.
    \item A GUI at the base-station is used to check the data and conditions as well as to send queries.
     \end{itemize}}
     
\project
	{Smart Library Return/Issue System}{Oct'19 - Dec'19}
	{}
	{\begin{itemize}
	\setlength\itemsep{0.3em}
    \item Worked with an STM32 + ESP8266 configuration to issue and return books.
    \item An RFID scanner was used to scan members IDs and books, and the database was queried wirelessly through the ESP8266.
    \item An LCD touchscreen  present on the development board was used to display information and communicate with the user.
     \end{itemize}}
     
\project
	{Using LPC2378 as a Music Player}{Sep'19 - Oct'19}
	{}
	{\begin{itemize}
	\setlength\itemsep{0.3em}
    \item A SD/MMC card containing wav files was to be read using MCI in LPC2378.
    \item LPC2378's hardware was used for playing the song, display and volume control.
    \item There were also buttons for play/pause, stop, next/previous song and forward/backward by 1/5 seconds.
     \end{itemize}}
     
\project
    {Voice Digitizer}{Mar'19 - April'19}
    {}
    {\begin{itemize}
	\setlength\itemsep{0.3em}
    \item This project was mainly focused on interfacing the Intel 8086 processor with various peripherals, to create a voice digitizer.
    \item It involved I/O mapping, memory mapping, clock generation, and the use of Interrupts. The system was designed using the MASM Assembly language and simulated on Proteus software.
     \end{itemize}}
 \end{projects}    
\vspace{-3mm}

\newpage

% \project
% 	{Interactive Neural Machine Translation (INMT)}{Jan'19 - Present}
% 	{
% 	    \textit{Advisors:  \href{https://www.microsoft.com/en-us/research/people/kalikab/}{Dr. Kalika Bali}, \href{https://www.microsoft.com/en-us/research/people/monojitc/}{Dr. Monojit Choudhury}, \href{https://www.microsoft.com/en-us/research/people/monojitc/}{Dr. Sandipan Dandapat}, \href{https://www.microsoft.com/en-us/research/people/monojitc/}{Tanuja Ganu}}
% 	}
% 	{\begin{itemize}
% 	\setlength\itemsep{0.3em}
%      \item Worked on understanding how translators can be assisted with suggestions from a machine translation system. On basis of the insights gathered, developed an interactive translation interface to make the translation process quicker and better in terms of quality. ~ [\href{https://microsoft.github.io/inmt/}{\small{\websiteSymbol}}] ~ {\small{\lbrack\textbf{{Demo@EMNLP'19}}\rbrack}}
%      \item Engaging with non-profits \href{https://translatorswithoutborders.org/}{Translators without Borders}, Pratham Books' \href{https://storyweaver.org.in/}{Story Weaver} and \href{http://cgnetswara.org/}{CGNet Swara} (covered by \href{https://www.livemint.com/mint-lounge/features/now-a-unique-machine-translation-tool-from-hindi-to-gondi-11597386377981.html}{LiveMint}) looking at possible solutions for deploying INMT for low resource languages. ~ {\small{\lbrack\textbf{{In Submission}}\rbrack}}
%      \item Developing new interfaces to tailor to specific use-cases of translation such as document translation and web-page localization (using \href{github.com/microsoft/inmt-browser}{browser extension}) and offline translation (\href{https://github.com/microsoft/INMT-lite}{INMT lite}).
%      \end{itemize}}
     