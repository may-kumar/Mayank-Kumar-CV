\sectionTitle{Projects}{}
\begin{projects}

\project
	{Building a Wireless Sensor Network (WSN) for Precision Agriculture}{Jan'20 - May'20}
	{\href{https://drive.google.com/file/d/1wIB6XobKUe-Rm_h03fx6rXY9kHtLBq16/view?usp=sharing}{(Project Report \pdfFileSymbol )}}
	{\begin{itemize}
	\setlength\itemsep{0.1em}
    \item Various motes were used together to make a WSN for monitoring and tracking growth of a crop.
    \item The network stack was developed to make the network as dynamic, long-lasting and robust as possible.
    \item A GUI at the base-station was used to check the data and conditions as well as to send queries.
     \end{itemize}}
     
\project
	{Smart Library Return/Issue System}{Oct'19 - Dec'19}
	{}
	{\begin{itemize}
	\setlength\itemsep{0.1em}
    \item Worked with an STM32 + ESP8266 configuration to issue and return books.
    \item An RFID scanner was used to scan members IDs and books, and the database was queried wirelessly through the ESP8266.
    \item An LCD touchscreen  present on the development board was used to display information and communicate with the user.
     \end{itemize}}
     
\project
	{Using LPC2378 as a Music Player}{Sep'19 - Oct'19}
	{\href{https://github.com/may-kumar/music-player_LPC-2378}{(Project Link \githubSymbol )}}
	{\begin{itemize}
	\setlength\itemsep{0.1em}
    \item A SD/MMC card containing wav files was to be read using MCI in LPC2378.
    \item LPC2378's hardware was used for playing the song, display and volume control.
    \item There were also buttons for play/pause, stop, next/previous song and forward/backward by 1/5 seconds.
     \end{itemize}}
     
\project
    {Voice Digitizer}{Mar'19 - April'19}
	{\href{https://github.com/may-kumar/voice-digitizer}{(Project Link \githubSymbol )}}
    {\begin{itemize}
	\setlength\itemsep{0.1em}
    \item This project was mainly focused on interfacing the Intel 8086 processor with various peripherals, to create a voice digitizer.
    \item It involved I/O mapping, memory mapping, clock generation, and the use of Interrupts. The system was designed using the MASM Assembly language and simulated on Proteus software.
     \end{itemize}}
 \end{projects}    
\vspace{-3mm}
